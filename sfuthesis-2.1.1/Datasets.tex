\chapter{Datasets}

In the majority of studies that can be found in the literature, the presented models for drug target interaction prediction are trained and evaluated on binary datasets. Typically the existing models are evaluated on the four datasets that were first presented in \cite{yamanishi2010drug}. In these datasets a label of $y_{d_i, t_j} = 1$ is given for a drug-target pair $(d_i, t_j)$ which is known to interact and a label of $y_{d_i, t_j} = 0$ is given when either the drug-target pair is known not to interact or when it is unknown whether the pair interacts. In contrast to a model that classifies if a drug-target pair interacts or not, the model that is developed in this thesis predicts the continuous binding affinity of drug target pairs. To the best of my knowledge, only one existing study can be found in the literature which presents a model for the prediction of the continuous binding affinity of drugs and targets \cite{pahikkala2014toward}. This study utilizes two continuous datasets (\textit{Metz} and \textit{Davis}) that are also used in this thesis to evaluate and compare the developed model. A third dataset \textit{KIBA} is obtained by preprocessing the drug-target dataset that is presented in \cite{tang2014making}.
The three datasets named \textit{Metz}, \textit{Davis} and \textit{KIBA} respectively that are used for evaluating the developed model and for comparing it to the model presented in \cite{pahikkala2014toward} are described in the following chapters. Additionally, each section describes the corresponding drug-drug and target-target similarity matrices that were used to construct the graphical model of the CCRF.

\section{The Davis Dataset}

The continuous dataset \textit{Davis} was used for the evaluation of the drug-target interaction prediction model presented in \cite{pahikkala2014toward}. The dataset itself was published in the study \cite{davis2011comprehensive}. For this dataset, the interaction of 72 kinase inhibitors with 442 kinases was tested and measured as the $K_d$ value. The kinase inhibitors are the drugs and the kinases are the targets in the more general formulation of drugs and targets. The \textit{Davis} dataset contains the full information of binding affinities for all drug-target pairs in the dataset, and thus contains no missing values. The binding affinities are measured as $K_d$ values. A lower $K_d$ value represents a higher binding affinity between the drug and the target. As described in \cite{davis2011comprehensive}, the binding affinity is not reported if it was measured to be $>10000$. For these drug target pairs, a $K_d$ value of $10000$ was imputed for the experiments in this thesis.
The $K_d$ values in the \textit{Davis} dataset were log transform, according to the formula:

\begin{equation}
pK_d:= -log_{10}(\frac{K_d}{1e9})
\end{equation}

\begin{figure}
\begin{center}
\includegraphics[scale=0.6]{davis_dist.png}
\end{center}
\caption{Distribution of $K_d$ values given in the Davis dataset.}
\label{fig:numStructure}
\end{figure}

\section{The Metz Dataset}

Just as the \textit{Davis} dataset, the continuous dataset \textit{Metz} was used for the evaluation of the drug-target interaction prediction model presented in \cite{pahikkala2014toward}. The dataset was published in the study \cite{metz2011navigating}. The \textit{Metz} dataset consists of 1421 drugs and 156 targets. The binding affinity is given as log transformed $K_i$ values for 42$\%$ of the drug-target pairs. 

\begin{figure}
\begin{center}
\includegraphics[scale=0.6]{metz_dist.png}
\end{center}
\caption{Distribution of $K_i$ values given in the Metz dataset.}
\label{fig:numStructure}
\end{figure}

\section{The KIBA Dataset}

The \textit{Davis} and \textit{Metz} datasets are suitable for the evaluation of predictive models for drug target interaction because data heterogeneity is not an issue. We can assume that the experimental settings for the measured drug target pairs in each dataset were the same and the binding affinities are comparable. When working with experimental results that come from multiple sources the data might be heterogeneous: In one case the binding affinity might be measured by $K_i$, in another case by $K_d$ and in a third case by $IC_{50}$ value. Another source of data heterogeneity are different experimental settings. An approach to integrate the observations from different sources, named \textit{KIBA} and a corresponding dataset is presented in \cite{tang2014making}. With their method, the authors of \cite{tang2014making} integrated the experimental results from multiple databases into a bioactivity matrix of 52498 compounds and 467 targets, including 246088 observations. The binding affinities in this matrix are given as \textit{KIBA}-scores. This dataset was used to obtain a third evaluation dataset, which is called the \textit{KIBA} dataset, by removing all drugs and targets with less than 10 observations from the original dataset that was downloaded from the supplementary material of \cite{tang2014making}, resulting in a dataset of 2116 drugs and 229 targets with a density of $~24\%$.

\begin{figure}
\begin{center}
\includegraphics[scale=0.6]{kiba_dist.png}
\end{center}
\caption{Distribution of KIBA scores given in the KIBA dataset.}
\label{fig:numStructure}
\end{figure}