\documentclass{sfuthesis}

\title{Continuous Conditional Random Fields for Drug Target Interaction Prediction}
\thesistype{Master}
\author{Marten Heidemeyer}
\previousdegrees{%
	B.Sc., Universit{\"a}t zu L{\"u}beck, 2013}
\degree{Master of Science}
\discipline{Computing Science}
\department{Department of Computing Science}
\faculty{}
\copyrightyear{2016}
\semester{Fall 2016}
\date{1 September 2015}

\keywords{Drug-target interaction, predictive modeling, machine learning}

\committee{%
	\chair{Faraz Harach}{}
	\member{}{Prof. Martin Ester\\Professor}
	\member{}{Dr. Artem Cherkasov\\Professor}
	\member{}{Leonid Chindelevitch\\Assistant Professor\\School of Computing Science}
}

%   PACKAGES AND CUSTOMIZATIONS  %%%%%%%%%%%%%%%%%%%%%%%%%%%%%%%%%%%%%%%%%%%%%%
%
%   Add any packages or custom commands you need for your thesis here.
%   You don't need to call the following packages, which are already called in
%   the sfuthesis class file:
%
%   - appendix
%   - etoolbox
%   - fontenc
%   - geometry
%   - lmodern
%   - nowidow
%   - setspace
%   - tocloft
%
%   If you call one of the above packages (or one of their dependencies) with
%   options, you may get a "Option clash" LaTeX error. If you get this error,
%   you can fix it by removing your copy of \usepackage and passing the options
%   you need by adding
%
%       \PassOptionsToPackage{<options>}{<package>}
%
%   before \documentclass{sfuthesis}.
%

\usepackage{amsmath,amssymb,amsthm}
\usepackage[pdfborder={0 0 0}]{hyperref}
\usepackage{graphicx}
\usepackage{caption}
\usepackage{tikz}
\usepackage{mathrsfs}
\usepackage{algpseudocode} 
\usepackage{algorithm}

\DeclareMathOperator*{\argmax}{argmax}






%   FRONTMATTER  %%%%%%%%%%%%%%%%%%%%%%%%%%%%%%%%%%%%%%%%%%%%%%%%%%%%%%%%%%%%%%
%
%   Title page, committee page, copyright declaration, abstract,
%   dedication, acknowledgements, table of contents, etc.
%

\begin{document}

\doublespacing
\large{
\frontmatter
\maketitle{}
\makecommittee{}

\begin{abstract}
Knowledge about the interaction between drugs and target proteins is essential in the drug discovery process.
Understanding the relationship between compounds and proteins through wetlab experiments alone is time-consuming and costly. With the motivation to support the experimental work by systematically prioritizing the most potent compounds for a target, numerous methods for the in-silico prediction of drug-target interaction have been proposed recently and high performance on binary datasets, where drug-target pairs are classified as either binding or non-binding have been reported. A possible drawback of binary datasets is that missing values and non interacting drug-target pairs are not differentiated. In this thesis, a model is developed that predicts the binding strengths of drug-target pairs as continuous values and thus incorporates the whole interaction spectrum from true negative to true positive interactions in the learning phase. The developed model combines two previously used approaches for the drug-target problem, which are Matrix Factorization and Conditional Random Fields. The model is evaluated in terms of the metrics $AUC$, $AUPR$ and $CI$ on three datasets and a slight improvement in performance is observered when compared to the state of the art method.
\end{abstract}

\begin{dedication} % optional
I dedicate this work to Mama, Papa and all my friends
\end{dedication}

\begin{acknowledgements} % optional
I would like to express my gratitude to my senior supervisor Prof. Martin Ester for his guidance, patience and support throughout my master studies. 
\end{acknowledgements}

\addtoToC{Table of Contents}\tableofcontents\clearpage
\addtoToC{List of Tables}\listoftables\clearpage
\addtoToC{List of Figures}\listoffigures
%   MAIN MATTER  %%%%%%%%%%%%%%%%%%%%%%%%%%%%%%%%%%%%%%%%%%%%%%%%%%%%%%%%%%%%%%
%
%   Start writing your thesis --- or start \include ing chapters --- here.
%

\mainmatter%


\chapter{Introduction}

\section{A Statement of the Problem}

Knowledge about the interaction strength between chemical structures and proteins is an important topic in drug development.
The goal in drug development is to find a chemical structure that binds to a diseases target protein without causing harmful side effects by binding to proteins other than the diseases target.
The safest and most accurate method to gain knowledge about the interaction strength of drug candidates and target proteins is through wetlab experiments. On the other hand, wetlab experiments are costly in terms of time and money, as there are thousands of potential drug candidates. The failure of a new ligand in toxicity tests is higher than $90\%$ which is the most significant reason for the high cost of the drug development process. In drug development, drug repositioning is a technique in which known drugs and drug candidates are used to treat new diseases. Existing drugs may bind to the target protein of a disease other than the disease that the drug was originally developed for. Using an existing drug as a basis for the development of a new drug is far more likely to succeed, because the existing drug has already passed toxicity tests and its safety is known. Numerous openly accessible databases exist, listing the interaction of known compounds, which can be either already approved drugs or experimental drug candidates, against known target proteins (ChEMBL \cite{gaulton2012chembl}, DrugBank \cite{wishart2008drugbank}, KEGG\cite{kanehisa2011kegg}, SuperTarget \cite{gunther2008supertarget}, BindingDB \cite{liu2007bindingdb}). The high cost of in vitro methods for the testing of drug-target binding behaviour and the availability of experimental results in public databases give strong incentives to develop in silico methods for the prediction of new drug target interactions.



\section{Review of Literature}

Machine Learning and Data Mining techniques for drug development is a hot topic. In most existing methods the problem is formulated as a binary classification problem, where the drug-target pairs are treated as instances and the chemical structures of drugs and the amino acid subsequences of the targets can are used as features, describing the instances. The goal in the binary formulation is to classify a given drug-target pair into binding and non binding. This approach stands in contrast with the developed method in this thesis, which predicts continuous drug target binding affinities. The related methods which classify drug target pairs will still be introduced here because the majority of existing work formulates the problem as such. The only existing method $KronRLS$ which predicts continuous binding affinities is introduced in the Methods section. The following sections describe the state of the art methods for drug target interaction prediction.

\subsection{Yamanishi, Yoshihiro, et al. "Prediction of drug-target interaction networks from the integration of chemical and genomic spaces." 2008}

One of the first proposed models for the task \cite{yamanishi2008prediction}, is a supervised bipartite graph learning method. The motivation of the model is to reveal the correlations between drug similarity, target similarity and the drug target interaction network. The authors define the \textit{chemical space} for drugs, the \textit{genomic space} for targets and the \textit{pharmacological space} for drug-target pairs and propose a method to embed compounds and proteins from the \textit{chemical} and \textit{genomic} spaces respectively into the unified \textit{pharmacological space}. New drug-target interactions are then predicted by connecting drugs and targets which are closer than a threshold in the \textit{pharmacological} space. 

In the authors model, the drug target interaction network is represented by a bipartite graph $G=(V_1+V_2, E)$ , where $V_1$ is a set of drugs, $V_2$ is a set of target proteins and $E$ is a set of interactions between the drugs and targets. A graph-based similarity matrix 
$ K = \begin{pmatrix}

K_{cc} & K_{cg} \\
K_{cg}^T & K_{gg}
\end{pmatrix}
$
is constructed, where the elements of $K_{cc}$, $K_{gg}$, $K_{cg}$ are computed by using Gaussian functions:
$(K_{cc})_{ij}=exp(-d^2_{c_i,c_j}/h^2)$ for $i,j=1,\dots,n_c$, $(K_{gg})_{ij}=exp(-d^2_{g_ig_j}/h^2)$ for $i,j=1,\dots,n_g$ and $(K_{cg})_{ij} = exp(-d^2_{c_ig_j}/h^2)$ for $i=1,\dots,n_c, j=1,\dots,n_g$. Here $d$ stands for the shortest distance between two objects,  $n_c$ and $n_g$ stand for the number of known drugs and targets respectively and $h$ is a width parameter. 
To compute the vectors that span the \textit{pharmacological} space, the eigenvalue decomposition of $K$ is computed as:

$K=\Gamma \Lambda^{\frac{1}{2}} \Lambda^{\frac{1}{2}} \Gamma ^T = UU^T$ and all drugs and targets are represented by using the row vectors of the matrix $U=(u_{c_1},\dots,u_{c_{n_c}},u_{g_1},\dots,u_{g_{n_g}})^T$ .

Now two models are learned to map new compounds and targets from the \textit{chemical} and \textit{genomic} space respectively into the \textit{pharmacological} space. A kernel regression model that learns the feature vectors of new compounds and targets is proposed for this task:
\begin{center}
$u=\sum\limits_{i=1}^{n}s(x,x_i)w_i+\epsilon$
\end{center}

for the mapping of the compounds, $s(x,x_i)$ represents the compound similarity and for the mapping of the targets $s(x,x_i)$ represents the target similarity. $\epsilon$ is a noise vector and $w_i$ is a weight vector that is learned by minimizing the loss function:
\begin{center}
$L=||UU^T - SWW^TS^T||^2_F$
\end{center}
where $S$ represents the respective similarity matrix, $W$ represents the matrix of weight vectors and $||.||_F$ represents the Frobenius norm.
Two such models, meaning two sets of weight vectors are learned to map new compounds and new targets onto the \textit{pharmacological} space.
Finally, based on the feature vectors $u$ in the \textit{pharmacological} space, feature-based similarity scores for three types of drug-target pairs are computed as the inner product as follows:
\begin{itemize}
\item $corr(c_{new}, g_j) = u_{cnew}u_{gj}$
\item $corr(c_i, g_{new}) = u_{ci}u_{gnew}$
\item $corr(c_{new},g_{new}) = u_{cnew}u_{gnew}$
\end{itemize}
High-scoring compound-protein pairs of any of the three above types are predicted to interact with each other.

\subsection{van Laarhoven, Twan, Sander B. Nabuurs, and Elena Marchiori. "Gaussian interaction profile kernels for predicting drug-target interaction." 2011}

The authors of this method first build an interaction profile for each drug and for each target. The interaction profile of each compound is a binary vector describing the presence or absence of interaction with every target in the network. The interaction profile for each target is defined analogously. The interaction profiles of the drugs and targets are used as feature vectors and a Gaussian kernel is constructed for the drugs and targets respectively. Let $y_{d_i}$ be the interaction profile of drug $d_i$, then the Gaussian kernel for the drugs is defined as
\begin{center}
$K_{GIP}(d_i,d_j) = exp(-\gamma_d || y_{di}-y_{dj} ||^2)$
\end{center}
and the Gaussian kernel for the targets can be defined analogously. Here, $GIP$ stands for Gaussian Interaction Profile and the parameter $\gamma_d$ controls the kernels bandwidth. The similarity information of the drugs and targets is integrated by defining two new kernels $K_{\text{chemical}}$ and $K_{\text{genomic}}$ for the drugs and targets respectively which are defined as:
\begin{center}
$S_{sym}=(S+S^T)/2$
\end{center}
where $S$ stands for the respective similarity matrix. Finally, a combined kernel of the two kernels above is defined as a weighted average:
\begin{center}
$K_d=\alpha_dK_{\text{chemical}}+(1-\alpha_d)K_{GIP}$\\
$K_t =\alpha_tK_{\text{genomic}}+(1-\alpha_t)K_{GIP}$
\end{center}
The prediction of the Regularized Least Squares (RLS) classifier for a given kernel $K$ is defined as:
\begin{center}
$\hat{y} = K(K+\sigma I)^{-1}y$
\end{center}
With the matrix $Y\in n_d \times n_t$ being the binary matrix of training values of $n_d$ drugs and $n_t$ targets and the Kernels for the drugs and targets as defined above, the authors of the model propose two ways to predict the interaction of all drug-target pairs in the matrix. The first type of prediction $RLS-avg$ is defined as:
\begin{center}
$\hat{Y}=\frac{1}{2}(K_d(K_d+\sigma I)^{-1}Y)+\frac{1}{2}(K_t(K_t+\sigma I)^{-1}Y^T)^T$
\end{center}
For the second type of prediction, the authors propose to compute yet a fourth kernel, defined by the Kronecker product $K = K_d \otimes K_t$, which gives a similarity for all drug-target pairs. The model named \textit{RLS-Kron} predicts $\hat{Y}$ as:
\begin{center}
$vec(\hat{Y}^T) = K(K+\sigma I)^{-1}vec(Y^T)$
\end{center}



%Two existing methods for drug target interaction prediction that are not based on machine learning techniques are docking simulation and ligand-based approaches. In docking simulation the interaction strength of ligands and proteins is estimated based on the structure of the target protein. This process is extremely time-consuming and the structural information of a protein is not always available \cite{liu2016neighborhood}. In ligand based approaches, the interaction strength of a candidate ligand to a target protein is obtained by comparing the candiate ligand to ligands for which the interaction strength to the target is known. This approach is not applicable, when information of candidate-similar ligands is not available for the target protein. Both approches will not be examined further here. 



\input{Method.tex}

\chapter{Experiments}

\section{Datasets}
\label{sec:datasets}
In the majority of studies that can be found in the literature, the presented models for drug-target interaction prediction are trained and evaluated on binary datasets. Typically, the existing models are evaluated on the four binary datasets that were first presented in \cite{yamanishi2010drug}. In these datasets a label of $y_{d_i, t_j} = 1$ is given for a drug-target pair $(d_i, t_j)$ which is known to interact and a label of $y_{d_i, t_j} = 0$ is given when either the drug-target pair is known not to interact or when it is unknown whether the pair interacts. In contrast to a model that classifies if a drug-target pair interacts or not, the model that is developed in this thesis learns and predicts the continuous binding affinities of drug-target pairs. To the best of my knowledge, only one existing study can be found in the literature which presents a model for the prediction of the continuous binding affinity of drugs and targets, which is the model $KronRLS$ \cite{pahikkala2014toward} which was introduced in section \ref{chp:review}. In the original paper, $KronRLS$ is evaluated on two continuous datasets (\textit{Metz} and \textit{Davis}) that are also used in this thesis for the evaluation of the presented model. A third dataset \textit{KIBA} is obtained by preprocessing the drug-target dataset that is presented in \cite{tang2014making}.
The three datasets named \textit{Metz}, \textit{Davis} and \textit{KIBA} respectively that are used for evaluating the developed model are described in the following chapters. Additionally, the corresponding drug-drug and target-target similarity matrices that were used to construct the graphical model of the CCRF are described in a following section. Table \ref{dataset_stats} lists the sizes and densities of the datasets. Here, density means the percentage of drug-target pairs in the dataset for which an observation is given.

\begin{table}[]
\centering
\begin{tabular}{l c c c}
Dataset & Drugs & Targets & Density \\
\hline
\textit{Davis} & 68 & 442 & 100\%\\
\textit{Metz} & 1421 & 156 & 42.1\%\\
\textit{KIBA} & 2116 & 229 & 24.4\%\\
\end{tabular}
\caption{Statistics of the used evaluation datasets.}
\label{dataset_stats}
\end{table}


\subsection{The Davis Dataset}

The continuous dataset \textit{Davis} was used for the evaluation of the drug-target interaction prediction model presented in \cite{pahikkala2014toward}. The dataset itself was published in the study \cite{davis2011comprehensive}. For this dataset, the interaction of 68 kinase inhibitors with 442 kinases was tested and measured as the $K_d$ value. The kinase inhibitors are the drugs and the kinases are the targets in the more general formulation of drugs and targets. The \textit{Davis} dataset contains the full information of binding affinities for all drug-target pairs in the dataset, and thus contains no missing values. A lower $K_d$ value indicates a higher binding affinity between the drug and the target. As described in \cite{davis2011comprehensive}, the binding affinity is not reported if it was measured to be $>10000$. For these drug target pairs, a $K_d$ value of $10000$ was used for the experiments in this thesis.
The $K_d$ values in the \textit{Davis} dataset were log transformed, according to the formula:

\begin{equation}
pK_d:= -\text{log}_{10}(\frac{K_d}{1e9})
\end{equation}

and thus after the log-transformation a higher $pK_d$ value represents a higher binding affinity. The drug-drug and target-target similarity matrices for this dataset can be downloaded from the website of the author of \cite{pahikkala2014toward}. The distribution of $pK_d$ values in this dataset is illustrated in figure \ref{fig:davis_dist}.

\begin{figure}
\begin{center}
\includegraphics[scale=0.6]{davis_dist.png}
\end{center}
\caption{Distribution of $pK_d$ values given in the \textit{Davis} dataset.}
\label{fig:davis_dist}
\end{figure}

\subsection{The Metz Dataset}

Just as the \textit{Davis} dataset, the continuous dataset \textit{Metz} was used for the evaluation of the drug-target interaction prediction model presented in \cite{pahikkala2014toward}. The dataset was published in the study \cite{metz2011navigating}. The \textit{Metz} dataset consists of 1421 drugs and 156 targets. The binding affinity is given as log transformed $K_i$ values (called $pK_i$ values) for 42$\%$ of the drug-target pairs. As drug-drug and target-target similarities for this dataset the matrices were used that can be downloaded from the website of the author of \cite{pahikkala2014toward}. The distribution of $pK_i$ values in this dataset is illustrated in figure \ref{fig:metz_dist}.
\begin{figure}
\begin{center}
\includegraphics[scale=0.6]{metz_dist.png}
\end{center}
\caption{Distribution of $K_i$ values given in the \textit{Metz} dataset.}
\label{fig:metz_dist}
\end{figure}

\subsection{The KIBA Dataset}

The \textit{Davis} and \textit{Metz} datasets are suitable for the evaluation of predictive models for drug target interaction because data heterogeneity is not an issue. We can assume that the experimental settings for the measured drug target pairs in each dataset were the same and the binding affinities are comparable. When working with experimental results that come from multiple sources the data might be heterogeneous: In one case the binding affinity might be measured by $K_i$, in another case by $K_d$ and in a third case by $IC_{50}$ value. Another source of data heterogeneity are different experimental settings. An approach to integrate the observations from different sources, named \textit{KIBA} (short for \textbf{K}inase \textbf{I}nhibitor \textbf{B}io\textbf{A}ctivity) and a corresponding dataset is presented in \cite{tang2014making}. With their method, the authors of \cite{tang2014making} integrated the experimental results from multiple databases into a bioactivity matrix of 52498 compounds and 467 targets, including 246088 observations. The binding affinities in this matrix are given as \textit{KIBA}-scores. This dataset was used to obtain a third evaluation dataset, which is called the \textit{KIBA} dataset, by removing all drugs and targets with less than 10 observations from the original dataset that was downloaded from the supplementary material of \cite{tang2014making}, resulting in a dataset of 2116 drugs and 229 targets with a density of $~24\%$. For this dataset the drug-drug similarity matrix was computed through the PubChem structure clustering tool (https://pubchem.ncbi.nlm.nih.gov/assay/assay.cgi?p=clustering). The target target similarity matrix was obtained by computing the normalized Smith Waterman Score \cite{yamanishi2010drug} for each pair of targets. The distribution of \textit{KIBA} scores in this dataset is illustrated in figure \ref{fig:kiba_dist}.

\begin{figure}
\begin{center}
\includegraphics[scale=0.6]{kiba_dist.png}
\end{center}
\caption{Distribution of KIBA scores given in the \textit{KIBA} dataset.}
\label{fig:kiba_dist}
\end{figure}

\subsection{Similarity Metrics of Drugs and Targets}
As drug-drug and target-target similarity matrices for the \textit{Davis} and \textit{Metz} dataset the precomputed matrices that are provided on the website of \cite{pahikkala2014toward} were used. Here, the drug-drug similarity was computed based on the 2D chemical structure of the compounds: The compounds are first represented by the graph of their 2D chemical structure and the similarity score is computed based on the size of the common substructures \cite{bajusz2015tanimoto}. The target-target similarity was computed based on the protein sequences, using the normalized Smith-Waterman score \cite{yamanishi2008prediction} . For the \textit{KIBA} dataset, the drug-drug similarity matrix was obtained through the compound clustering tool of PubChem. The given ChEMBL IDs of the compounds were first matched to their PubChem CIDs which were then used as input to the PubChem web interface (The PubChem website offers a tool that allows to match ChEMBL IDs to PubChem CIDs). The clustering tool allows to download a similarity matrix for the compounds which is computed based on the compound structure (similarly as for the drug-drug similarity of the \textit{Metz} and \textit{David} datasets). For the \textit{KIBA} dataset, the protein sequences were downloaded from \textit{NCBI} and the normalized Smith Waterman similarity was computed for each pair by aligning the sequences using the \textit{Biostrings} R package.

\subsubsection{Correlation of Drug/Target Similarity and Binding Behaviour}
In order to examine the correlation between the similarity matrices and the binding behaviour (meaning to find out if similar drugs or targets actually show similar binding behaviour) the following data analysis was performed. First, the analysis regarding the drug-similarity: all pairs of drugs were selected, for which five or more targets can be found, such that both drugs were tested against those five or more targets. The pairs of drugs were subset into two disjunct sets. The first subset contains only pairs of drugs for which the similarity is below a threshold of $0.7$, meaning that this subset contains drug pairs of low similarity. The second subset contains all the drug pairs for which the similarity is above the threshold, meaning this subset contains drug pairs of high similarity. The following steps were then performed for both subsets: For each pair of drugs $(d_a,d_b)$ in the set, all targets that both drugs were tested against were selected. This set of targets was subset into $binding_a$, representing all targets to which $d_a$ binds with a high affinity (the thresholds to define high affinity for the datasets are described in the next section) and $binding_b$, representing all targets to which $d_b$ binds with a high affinity. For each pair of drugs, the size of the union of $binding_a$ and $binding_b$ (x-Axis in the following plots) and the size of the intersection of $binding_a$ and $binding_b$ (y-Axis in the following plot) was computed. Finally, the number of times each combination of union and intersection was obtained is counted (the count is represented by the bubble size and color in the following plots). The same analysis can be repeated analogously for the similar and un-similar target pairs of the datasets.

Figures \ref{fig:metz_drug_simi_corr_1} and \ref{fig:metz_drug_simi_corr_2} show on the x-Axis the size of the union and on the y-Axis the size of the intersection of the drug-drug pairs for the \textit{Metz} dataset. The bubble-size represents the number of drug-drug pairs that were found with the corresponding sizes of the union and intersection. Comparing figures \ref{fig:metz_drug_simi_corr_1} and \ref{fig:metz_drug_simi_corr_2}, we can see that drug-drug pairs with a similarity above $0.7$ tend to share more targets to which both drugs bind with high affinities than drug-drug pairs with a similarity below $0.7$. The same can be observed for drug-drug pairs in the \textit{KIBA} dataset illustrated in figures \ref{fig:kiba_drug_simi_corr_1} and \ref{fig:kiba_drug_simi_corr_2}. For the \textit{Davis} dataset it seems like no such correlation can be observed (see figures \ref{fig:davis_drug_simi_corr_1} and \ref{fig:davis_drug_simi_corr_2}) which goes hand in hand with the observation in the results section, that the drug similarity does not improve the prediction performance. On the other hand as illustrated in figures \ref{fig:davis_target_simi_corr_1} and \ref{fig:davis_target_simi_corr_2}, the target similarity correlates with the binding behaviour of the targets for the \textit{Davis} datasets.


\begin{figure}[p]
\begin{center}
\includegraphics[scale=0.36]{metz_simi_above.png}
\end{center}
\caption[Correlation of drug-similarity and binding behaviour, similar drug pairs, Metz dataset]{Illustration of binding behaviour of drug-drug pairs in \textit{Metz} dataset  with similarity above $0.7$.}
\label{fig:metz_drug_simi_corr_1}
\begin{center}
\includegraphics[scale=0.36]{metz_simi_below.png}
\end{center}
\caption[Correlation of drug-similarity and binding behaviour, un-similar drug pairs, Metz dataset]{Illustration of binding behaviour of drug-drug pairs in \textit{Metz} dataset with similarity below $0.7$.}
\label{fig:metz_drug_simi_corr_2}
\end{figure}


\begin{figure}[p]
\begin{center}
\includegraphics[scale=0.36]{kiba_simi_above.png}
\end{center}
\caption[Correlation of drug-similarity and binding behaviour, similar drug pairs, \textit{KIBA} dataset]{Illustration of binding behaviour of drug-drug pairs in \textit{KIBA} dataset  with similarity above $0.7$.}
\label{fig:kiba_drug_simi_corr_1}
\begin{center}
\includegraphics[scale=0.36]{kiba_simi_below.png}
\end{center}
\caption[Correlation of drug-similarity and binding behaviour, un-similar drug pairs, \textit{KIBA} dataset]{llustration of binding behaviour of drug-drug pairs in \textit{KIBA} dataset  with similarity below $0.7$.}
\label{fig:kiba_drug_simi_corr_2}
\end{figure}

\begin{figure}[p]
\begin{center}
\includegraphics[scale=0.36]{davis_simi_above.png}
\end{center}
\caption[Correlation of drug-similarity and binding behaviour, similar drug pairs, \textit{Davis} dataset]{Illustration of binding behaviour of drug-drug pairs in \textit{Davis} dataset  with similarity above $0.7$.}
\label{fig:davis_drug_simi_corr_1}
\begin{center}
\includegraphics[scale=0.36]{davis_simi_below.png}
\end{center}
\caption[Correlation of drug-similarity and binding behaviour, un-similar drug pairs \textit{Davis} dataset]{Illustration of binding behaviour of drug-drug pairs in \textit{Davis} dataset  with similarity below $0.7$.}
\label{fig:davis_drug_simi_corr_2}
\end{figure}


\begin{figure}[p]
\begin{center}
\includegraphics[scale=0.36]{davis_targets_above.png}
\end{center}
\caption[Correlation of target-similarity and binding behaviour, similar target pairs, \textit{Davis} dataset]{Illustration of binding behaviour of target-target pairs in \textit{Davis} dataset with similarity above $0.7$.}
\label{fig:davis_target_simi_corr_1}
\begin{center}
\includegraphics[scale=0.36]{davis_targets_below.png}
\end{center}
\caption[Correlation of target-similarity and binding behaviour, un-similar target pairs, \textit{Davis} dataset]{Illustration of binding behaviour of target-target pairs in \textit{Davis} dataset with similarity below $0.7$.}
\label{fig:davis_target_simi_corr_2}
\end{figure}







\section{Experimental Settings}

\subsection{Integrating the similarity matrices}
For the MF+CCRF method that is developed in this thesis it is possible to integrate:
\begin{itemize}
\item none of the similarity matrices
\item only the drug drug similarity matrix
\item only the target target similarity matrix
\item both similarity matrices
\end{itemize}
Here, using no similarity metric means to use just the MF prediction. When only the drug-drug similarity metric is used, we build graphical models for each column (target) of the matrix. Thus a separate set of parameters is learned and independent predictions are made for each target. Similarly, if only the target-target similarity metric is used, a separate graphical model is build for each row (drug) of the matrix and separate paratemers are learned/independant predictions made for each drug. When both similarity matrices are used each cell of the matrix becomes a node of the CCRF and by default we would learn only one set of parameters for this CCRF. The datasets that were used for the experiments in this thesis are too large to create a single graphical model for all drugs and targets because in the inference step it is necessary to compute the inverse of a matrix of size $|D||T| \times |D||T|$. In order to integrate both similarity matrices for the CCRF the drug-target matrix was therefore clustered into smaller submatrices of feasible size. 


\section{Results}
\label{chapt:res}
This chapter lists the results of the experiments as described in the previous section for the comparison model $KronRLS$ and the \textit{MF+CCRF} model.
Tables \ref{res:ci}, \ref{res:auc} and \ref{res:aupr} list the performances in terms of $CI$, $AUC$ and $AUPR$ respectively for both methods and all three datasets. The cases of integrating either only one similarity matrix or both similarity matrix are listed separately as discussed above. For the \textit{MF+CCRF} model the performance of using only \textit{MF} is listed additionally. For the $KronRLS$ model the table entry where none of the similarities is used, is left empty. We observe the following:

\begin{itemize}
\item Regarding the performance of the \textit{MF+CCRF} model, we observe that the integration of the similarity matrices through the \textit{CCRF} brings a significant improvement in performance when compared to the performance of using only \textit{MF}. The improvement is most evident for the classification metrics $AUC$ and $AUPR$. For the \textit{Davis} dataset, the integration of the similarity matrices \textit{CCRF} raises the $AUC$ from $0.86$ (\textit{MF}) to $0.95$ (\textit{MF+CCRF}, both similarity matrices). For the \textit{Metz} dataset, the \textit{AUC} is raised from $0.88$ (\textit{MF}) to $0.94$ (\textit{MF+CCRF}, both similarity matrices). In terms of \textit{AUPR}, the \textit{CCRF} rises the performance from $0.49$ (\textit{MF}) to $0.67$ (\textit{MF+CCRF}, both similarity matrices) on the \textit{Davis} dataset, from $0.33$ (\textit{MF}) to $0.58$ (\textit{MF+CCRF}, both similarity matrices) on the \textit{Metz} dataset and from $0.63$ (\textit{MF}) to $0.72$ (\textit{MF+CCRF}, both similarity matrices) on the \textit{KIBA} dataset.
\item When comparing the performance of \textit{MF+CCRF} to $KronRLS$ in the settings where only one of the similarity matrices is used, we observe that \textit{MF+CCRF} significantly outperforms $KronRLS$ on most of the datasets. In particular for the \textit{KIBA} dataset, \textit{MF+CCRF} raises the $CI$ from $0.67$ to $0.80$ when only the drug similarity is used and from $0.65$ to $0.80$ when only the target similarity is used. On the \textit{Metz} dataset, \textit{MF+CCRF} raises the $CI$ from $0.74$ to $0.81$ when only the drug similarity is used and from $0.69$ to $0.78$ when only the target similarity is used. Further, \textit{MF+CCRF} raises the $AUC$ on the Metz dataset from $0.86$ to $0.91$, when using only the drug similarity and from $0.84$ to $0.90$, when using only the target similarity. On the \textit{Davis} dataset, \textit{MF+CCRF} raises the $AUC$ from $0.75$ to $0.81$ when using only the drug similarity.
\item When integrating both similarity metrics, we observe almost equal performances for $KronRLS$ and \textit{MF+CCRF} for across all datasets and metrics: \textit{MF+CCRF} marginally improves the performance of $KronRLS$ in regard of $CI$ for the \textit{Metz} dataset, where it raises the $CI$ of $0.78$ to $0.82$. In all other cases the performances of $KronRLS$ and \textit{MF+CCRF} are quite similar and do not differ by more than 3$\%$.
\end{itemize}



\begin{table}
\centering
\captionof{table}{$CI$ of $KronRLS$ and $MF+CCRF$ on the three evaluation datasets.}
\label{res:ci}
\resizebox{0.8\textwidth}{!}{
\begin{tabular}{l | c}
Dataset & CI \\
\hline
Davis &
\begin{tabular}{c c}
KronRLS & CCRF \\ \hline
\begin{tabular}{c|c|c}
2D & 0.85 & 0.70\\ \hline
$\delta$ & 0.84 &\\ \hline
 & SW & $\delta$\\
\end{tabular} & 
\begin{tabular}{c|c|c} 
2D & 0.88 & 0.73 \\ \hline
$\delta$ & 0.86 & 0.81\\ \hline
& SW & $\delta$ \\
\end{tabular} 
\end{tabular} \\
Metz & 
\begin{tabular}{c c}
KronRLS & CCRF \\ \hline
\begin{tabular}{c|c|c}
2D & 0.78 & 0.74   \\ \hline
$\delta$ & 0.69 & \\ \hline
 & SW & $\delta$\\
\end{tabular} & 
\begin{tabular}{c|c|c} 
2D & 0.82 & 0.81  \\ \hline
$\delta$ & 0.78 & 0.78\\ \hline
& SW & $\delta$ \\
\end{tabular} 
\end{tabular} \\
Kiba& 
\begin{tabular}{c c}
KronRLS & CCRF \\ \hline
\begin{tabular}{c|c|c}
2D & 0.79 & 0.67\\ \hline
$\delta$ & 0.65 & \\ \hline
 & SW & $\delta$\\
\end{tabular} & 
\begin{tabular}{c|c|c} 
2D & 0.81 & 0.80\\ \hline
$\delta$ & 0.80 & 0.79 \\ \hline
& SW & $\delta$ \\
\end{tabular} 
\end{tabular}\\
\end{tabular}}
\end{table}


\begin{table}
\centering
\captionof{table}{$AUC$ of $KronRLS$ and $MF+CCRF$ on the three evaluation datasets.}
\label{res:auc}
\resizebox{0.8\textwidth}{!}{
\begin{tabular}{l | c}
Dataset & AUC\\
\hline
Davis & 
\begin{tabular}{c c}
KronRLS & CCRF \\ \hline
\begin{tabular}{c|c|c}
2D & 0.93 & 0.75 \\ \hline
$\delta$ & 0.91  &  \\ \hline
 & SW & $\delta$\\
\end{tabular} & 
\begin{tabular}{c|c|c} 
2D & 0.95 & 0.81\\ \hline
$\delta$ & 0.93 &  0.86\\ \hline
 & SW & $\delta$ \\
\end{tabular} 
\end{tabular}  
 \\
Metz & 
\begin{tabular}{c c}
KronRLS & CCRF \\ \hline
\begin{tabular}{c|c|c}
2D & 0.93 & 0.86 \\ \hline
$\delta$ & 0.84 & \\ \hline
 & SW & $\delta$\\
\end{tabular} & 
\begin{tabular}{c|c|c} 
2D & 0.94 & 0.93\\ \hline
$\delta$ & 0.90 & 0.88\\ \hline
 & SW & $\delta$ \\
\end{tabular} 
\end{tabular} \\
Kiba& 
\begin{tabular}{c c}
KronRLS & CCRF \\ \hline
\begin{tabular}{c|c|c}
2D & 0.88 & 0.85\\ \hline
$\delta$ & 0.84 & \\ \hline
 & SW & $\delta$\\
\end{tabular} & 
\begin{tabular}{c|c|c} 
2D & 0.85 & 0.83\\ \hline
$\delta$ & 0.83 &0.83 \\ \hline
 & SW & $\delta$ \\
\end{tabular} 
\end{tabular}  \\
\end{tabular}}\\
\end{table}

\begin{table}
\centering
\captionof{table}{$AUPR$ of $KronRLS$ and $MF+CCRF$ on the three evaluation datasets.}
\label{res:aupr}
\resizebox{0.8\textwidth}{!}{
\begin{tabular}{l | c}
Dataset & AUPR \\
\hline
Davis &
\begin{tabular}{c c}
KronRLS & CCRF \\ \hline
\begin{tabular}{c|c|c}
2D & 0.68 & 0.27\\ \hline
$\delta$ & 0.64 &\\ \hline
 & SW & $\delta$\\
\end{tabular} & 
\begin{tabular}{c|c|c} 
2D & 0.67 & 0.34\\ \hline
$\delta$ & 0.60  & 0.49\\ \hline
 & SW & $\delta$ \\
\end{tabular} 
\end{tabular} \\
Metz & 
\begin{tabular}{c c}
KronRLS & CCRF \\ \hline
\begin{tabular}{c|c|c}
2D &  0.57 &  0.44 \\ \hline
$\delta$ & 0.28 & \\ \hline
 & SW & $\delta$\\
\end{tabular} & 
\begin{tabular}{c|c|c} 
2D & 0.58 & 0.55\\ \hline
$\delta$ & 0.43 & 0.33\\ \hline
 & SW & $\delta$ \\
\end{tabular} 
\end{tabular} \\
Kiba & 
\begin{tabular}{c c}
KronRLS & CCRF \\ \hline
\begin{tabular}{c|c|c}
2D & 0.75 & 0.67 \\ \hline
$\delta$ & 0.66 & \\ \hline
 & SW & $\delta$\\
\end{tabular} & 
\begin{tabular}{c|c|c} 
2D & 0.72 & 0.68\\ \hline
$\delta$ & 0.69 & 0.63 \\ \hline
 & SW & $\delta$ \\
\end{tabular} \\
\end{tabular}\\
\end{tabular}}\\
\end{table}

\chapter{Discussion and Future Work}

\section{Discussion}

In this work, a new method, \textit{MF+CCRF}, for the prediction of drug target interaction, which combines Matrix Factorization and Continuous Conditional Random Fields is proposed. In contrast to the vast majority of previous work on machine-learning based methods for drug-target interaction prediction that classify drug-target pairs as either binding or non-binding the proposed model predicts the binding affinity as continuous values, which better reflects the true complexity of the drug-target prediction problem. The model is evaluated on three datasets which are arguably of especially high quality in terms of data homogeneity as the first two data sets, \textit{Davis} and \textit{Metz}, originate from individual wetlabs, where the measurements were taken under unified experimental conditions and for the third dataset, \textit{KIBA}, observations from multiple sources where carefully integrated. Rather than containing only true positive interactions as the previously used binary datasets, the datasets that were used here
contain standardized mappings of the $K_i$, $K_d$ and $KIBA$-scores which provide broader insights into the interaction patterns of the drugs and their potential protein targets. The obtained results in terms of the ranking metric $CI$ and the classification metrics $AUC$ and $AUPR$ show that the proposed methods performs as well as the sate of the art method $KronRLS$. As described in section \ref{bincont}, $KronRLS$ predicts binary values when the classification metrics $AUC$ and $AUPR$ are applied, while $MF+CCRF$ predicts continuous values on which the binarization threshold is applied after the prediction step. $MF+CCRF$ still performs as good as $KronRLS$ in terms of $AUC$ and $AUPR$ and therefore one can argue that $MF+CCRF$ has the advantage that for different binarization thresholds, $MF+CCRF$ does not need to be retrained, in contrast to $KronRLS$. Especially for the cases when only one similarity matrix of either the drugs or the targets is given, $MF+CCRF$ outperforms $KronRLS$ significantly across almost all datasets and metrics. 

Further, we observe that the integration of the similarity matrices through the $CCRF$ significantly improves the performance of Matrix Factorization alone. This suggests that the model can be used to improve the $MF$ prediction in similar settings, for example in user-item recommendation tasks, when suitable similarity matrices for the users/items are given.


\section{Future Work}
One issue that is not addressed in this work is the problem of the biased nature of drug-target datasets. The used evaluation datasets are usually highly biased, containing only a small number of drugs and targets with many observations and a large number of drugs and targets with only a few observations. In the cross validation setting this leads to overoptimistic results because the model is mainly evaluated on the few drugs and targets with many observations and the reported evaluation metrics can not be generalized to the complete set of drugs and targets. One direction of future work would be to provide researchers a measure of the confidence of the prediction, so that the most confident predictions can be validated in wetlab experiments.

An other direction of future work would be to find a better strategy for the parameter tuning of the presented model. $KronRLS$ uses an inner cross validation step to find the optimal regularization parameter $\lambda$. A similar strategy could be applied here, where the $MF+CCRF$ model could automatically search for drug-similarity and target-similarity thresholds, that define which nodes of the CCRF to connect, instead of using the fixed parameter setting of connecting all $k$-nearest neighbors.




%   BACK MATTER  %%%%%%%%%%%%%%%%%%%%%%%%%%%%%%%%%%%%%%%%%%%%%%%%%%%%%%%%%%%%%%
%
%   References and appendices. Appendices come after the bibliography and
%   should be in the order that they are referred to in the text.
%
%   If you include figures, etc. in an appendix, be sure to use
%
%       \caption[]{...}
%
%   to make sure they are not listed in the List of Figures.
%

\backmatter%
	\addtoToC{Bibliography}
	\bibliographystyle{plain}
	\bibliography{references}

\end{document}
